\chapter{\trilingual{Conclusion}{Besluit}{Conclusion}}
The variable $E$ is known as the energy\nomenclature{$E$}{Energy}. In \textbf{Options/Configure TeXStudio/Commando's} you have to modify \textbf{Makeglossaries} with \texttt{"c:\textbackslash Program Files\textbackslash MiKTeX 2.9\textbackslash miktex\textbackslash bin\textbackslash x64\textbackslash makeindex" \%.nlo -s nomencl.ist -o \%.nls} to run \textbf{Tools/Glossary}.
\begin{equation}
	E = mc^2
\end{equation}
A reference \cite{lauwens2010performance} ... Do not forget to run \textbf{Tools/Bibliography}.
See tabel \ref{tabletest}
\begin{lstlisting}[language=julia]
#= This is a code sample for the Julia language
(adapted from http://julialang.org) =#
function mandel(z)
c = z
maxiter = 80
for n = 1:maxiter
	if abs(z) > 2
		return n-1
	end
	z = z^2 + c
end
return maxiter
end

function helloworld()
println("Hello, World!") # Bye bye, MATLAB!
end

function randmatstat(t)
n = 5
v = zeros(t)
w = zeros(t)
for i = 1:t
	a = randn(n,n)
	b = randn(n,n)
	c = randn(n,n)
	d = randn(n,n)
	P = [a b c d]
	Q = [a b; c d]
	v[i] = trace((P.'*P)^4)
	w[i] = trace((Q.'*Q)^4)
end
std(v)/mean(v), std(w)/mean(w)
end
\end{lstlisting}
\begin{figure}
	\centering
	\includegraphics[width=3cm]{logo-rma}
	\caption{The new RMA logo.}
\end{figure}
\begin{table}
	\centering
	\begin{tabular}{||c c c c||} 
		\hline
		Col1 & Col2 & Col2 & Col3 \\ [0.5ex] 
		\hline\hline
		1 & 6 & 87837 & 787 \\ 
		2 & 7 & 78 & 5415 \\
		3 & 545 & 778 & 7507 \\
		4 & 545 & 18744 & 7560 \\
		5 & 88 & 788 & 6344 \\ [1ex] 
		\hline
	\end{tabular}
	\caption{Table to test captions and labels.}
	\label{tabletest}
\end{table}