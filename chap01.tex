\chapter{\trilingual{Introduction}{Inleiding}{Introduction}}
\section{\trilingual{Contexte}{Achtergrond}{Background}}

Cette thèse fait partie d'un grand projet "Intelligent Recognition Information System (IRIS)" qui est pris en charge par le département CISS de l'école royale militaire. Le projet a débuté en janvier 2019 et a pour de développer des outils afin d'aider l'équipage des véhicules blindés à exécuter leurs tâches (engager ou non un ennemi potentiel, faire de la reconnaissance ...). Le projet se compose de trois grandes étapes : 
\begin{itemize}
    \item La détection et classification d'objets au sol grâce à des capteurs situés à l'avant du véhicule. Cela conduit à la création d'une grande collection de données sur les objets rencontrés sur le terrain et leurs liens avec le véhicule militaire (position, distance, etc.). {à citer SAHARA
Semi-Automatic Help for Aerial Region Analysis}
    \item La détection des différentes menaces potentielles afin de créer une carte de ce qui est connu sur le terrain.
    \item La création d'une stratégie d'attaque à partir de la carte de situation afin de traiter les différentes menaces.
\end{itemize}

Dans le cadre du troisième point présenté ci-dessus, une partie a été faite par le Cdt Koen BOECKX, ir et qui sera reprise et étudiée en détail. Son travail consiste en : 
\begin{itemize}
    \item Premièrement, le développement d'un modèle algorithmique représentant le terrain. Les acteurs tels que les forces amies et ennemies peuvent interagir avec l'environnement, par exemple en tirant, en se déplaçant, en visant... . L'environnement contient des obstacles qui, par leur présence, peuvent restreindre la visibilité et la mobilité des agents. 
    \item Deuxièmement, voir s'il est possible de créer une stratégie d'attaque basée sur des algorithmes de multi-agents basées sur l'intelligence artificielle et des théorie comme "l'apprentissage approfondi (Deep Learning)" et "l'apprentissage par Renfoncement (Reinforcement Learning)" .
\end{itemize}







\section{\trilingual{Analyse de la littérature}{Literatuurstudie}{Literature Review}}
\subsection{L'Apprentissage par Renforcement}

L'idée derrière l'apprentissage par renforcement est d'apprendre en interagissant avec l'environnement. Ce type d'apprentissage est celui que les humains expérimentent dès la naissance. Un enfant après la naissance n'a pas d'enseignant explicite. Il dispose de ses différents sens qui lui permettent d'obtenir des informations sur son environnement. \\ 

Le domaine de "l'apprentissage automatique" peut être divisé en deux grandes catégories, dont il faut comprendre la différence. Ces deux catégories sont "l'Apprentissage Supervisé (Supervised Learning)" et "l'apprentissage par renforcement". \\ 

L'apprentissage supervisé consiste à construire une fonction qui associe une certaine entrée à une certaine sortie sur la base d'exemples.  Ces exemples sont étiquetés en fonction de la sortie souhaitée. L'objectif est d'interpréter correctement les nouveaux exemples. \\ 

Contrairement au dernier type d'apprentissage, dans l'apprentissage par renforcement, l'agent s'entraîne par l'expérience, non pas en lui donnant une série d'exemples mais en le mettant dans l'environnement. Plus de détails vont suivre dans le paragraphe suivante. 

\subsubsection{Les éléments de l'apprentissage par renforcement \cite{Lapan18}}

Il existe deux entités majeures dans l'apprentissage par le renforcement.  Il s'agit de l'environnement et de l'action. 













