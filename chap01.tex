\chapter{\trilingual{Introduction}{Inleiding}{Introduction}}
\section{\trilingual{Contexte}{Achtergrond}{Background}}

Cette thèse fait partie d'un grand projet "Intelligent Recognition Information System (IRIS)" qui est pris en charge par le département CISS de l'école royale militaire. Le projet a débuté en janvier 2019 et a pour de développer des outils afin d'aider l'équipage des véhicules blindés à exécuter leurs tâches (engager ou non un ennemi potentiel, faire de la reconnaissance ...). Le projet se compose de trois grandes étapes : 
\begin{itemize}
    \item La détection et classification d'objets au sol grâce à des capteurs situés à l'avant du véhicule. Cela conduit à la création d'une grande collection de données sur les objets rencontrés sur le terrain et leurs liens avec le véhicule militaire (position, distance, etc.). {à citer SAHARA
Semi-Automatic Help for Aerial Region Analysis}
    \item La détection des différentes menaces potentielles afin de créer une carte de ce qui est connu sur le terrain.
    \item La création d'une stratégie d'attaque à partir de la carte de situation afin de traiter les différentes menaces.
\end{itemize}

Dans le cadre du troisième point présenté ci-dessus, une partie a été faite par le Cdt Koen BOECKX, ir et qui sera reprise et étudiée en détail. Son travail consiste en : 
\begin{itemize}
    \item Premièrement, le développement d'un modèle algorithmique représentant le terrain. Les acteurs tels que les forces amies et ennemies peuvent interagir avec l'environnement, par exemple en tirant, en se déplaçant, en visant... . L'environnement contient des obstacles qui, par leur présence, peuvent restreindre la visibilité et la mobilité des agents. 
    \item Deuxièmement, voir s'il est possible de créer une stratégie d'attaque basée sur des algorithmes de multi-agents basées sur l'intelligence artificielle et des théorie comme "l'apprentissage approfondi (Deep Learning)" et "l'Apprentissage par Renfocement (Reinfocement Learning)" .
\end{itemize}






\section{\trilingual{Analyse de la littérature}{Literatuurstudie}{Literature Review}}